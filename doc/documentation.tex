\documentclass{article}
\usepackage{listings} \lstset{numbers=left, numberstyle=\tiny, numbersep=5pt} 
\usepackage{tikz}
\usetikzlibrary{arrows,shadows}
\usepackage{pgf-umlsd}

\begin{document}
\title{GDB Library for client remote connection}
\author{Rene L\"ammert}

\maketitle

\begin{abstract}
libgdbc ist eine library welche die Verbindung bzw Kommunikation mit einem gdbserver verwalten soll.
Hauptziel dabei ist es die bestehende Implementierung im radare2 Projekt zu ersetzen.
\end{abstract}

\section{Ben\"otigt}
Dieser Abschnitt beschreibt die Anforderungen an die Funktionalit\"at der Bibliothek.


\subsection{Funktionen}
\begin{descrition}
	\item[Connect]\hfill \\
	\item[Disconnect]\hfill \\
	\item[Stepping]\hfill \\
	\item[Read Register]\hfill \\
	\item[Write Register]\hfill \\
	\item[Read Memory]\hfill \\
	\item[Write Memory]\hfill \\
\end{description}
\subsection{Allgemeines}
\begin{description}
	\item[Useflag style um radare, windows und linux versionen bauen zu k\"onnen]\hfill \\
	\item[In erster Linie bezogen auf Sockets]\hfill \\
\end{description}
\subsection{Tests}
\begin{description}
	\item[Komplette debug session]\hfill \\
		\input{rebalance}
	\item[Qemu]\hfill \\
	\item[Linux]\hfill \\
	\item[Windows]\hfill \\
	\item[VM-Ware]\hfill \\
	\item[automatische tests aller funktionen]\hfill \\
\end{description}

\subsection{Aufbau}
\begin{description}
	\item[libgdbbc]\hfill \\
	Haupteinstiegspunkt in die lib. Die Instanz der gesamten lib wird hier verborgen.
	\item[core]\hfill \\
	\item[messages]\hfill \\
	\item[packet]\hfill \\
	\item[utils]\hfill \\
\end{description}
\end{document}
